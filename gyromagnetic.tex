\documentclass[a4paper]{article}
\usepackage{epsfig} 
\usepackage{amsmath}
\usepackage{amsfonts}
\usepackage{graphicx}
%\usepackage{mathtools}
\usepackage{cite}
\usepackage{float}
%\usepackage{tipa}

\title{Gyromagnetic ratios of black holes}
\author{Helgi Freyr R\'{u}narsson}
\begin{document}
\maketitle

The gyromagnetic ratio of a charged rotating object is the ratio of its magnetic dipole moment to its angular momentum.
For an electron, it is equal to two up to loop quantum corrections.
For four dimensional Kerr-Newman black holes, it is also equal to two\cite{Carter:1968rr}.
The same value is found for so-called Kerr-Sen black holes\cite{Sen:1992ua} and some $p$-brane solutions in eleven dimensions\cite{Balasubramanian:1998za}.
In contrast to this seemingly universal value, the gyromagnetic ratio of Kaluza-Klein black holes depends strongly on the parameters of the metric\cite{Larsen:1999pp}.
Supergravity Dirichlet $0$-brane solutions in ten dimensions have been shown to have $g=1$\cite{Ardalan:1998ce}.

However, as these deviations from the value of $g=2$ are present in higher dimensions or other theories of gravity, it is of interest to investigate whether four-dimensional classical general relativity solutions can show different values of the gyromagnetic ratio.
In 1967, Ernst put forth a prescription on how to derive axially symmetric stationary solutions of Einstein's equations\cite{Ernst:1967wx,Ernst:1967by}.
This prescription allowed Ernst derive solutions found by Weyl\cite{Weyl:1917gp} and Papapetrou\cite{Papapetrou:1953zz} as well as both the Kerr and Kerr-Newman solutions in a relatively simple manner.
In fact, Tomimatsu and Sato found a solution by this prescription which reduces to the Weyl solution in the limit of $J=0$ and the Kerr solution when $J=m^2$, where $J$ and $m$ are the angular momenta and mass of the solution respectively.\cite{Tomimatsu:1972zz}
Ernst then found a charged version of this solution\cite{PhysRevD.7.2520}.

It was then shown by Reina and Treves that any electrovac solution generated by the Ernst prescription, has a gyromagnetic ratio of two.\cite{Reina:1975rt}

In four dimensions, Kerr-Newman-AdS black holes have been shown to have a gyromagnetic ratio of $2$\cite{Aliev:2007qi} while in higher dimensions when the black hole has a single angular momenta, $g$ depends on the dimensionless ratio of the rotation parameter and the curvature radius of the background.
In five dimensions with two angular momenta, the black hole possesses two distinct gyromagnetic ratios corresponding to the two $2$-planes of rotation.
When the two angular momenta are equal, the two gyromagnetic ratios merge and the black hole has $g=4$\cite{Aliev:2007qi}.

In Fig.~\ref{fig:gyro}, we see the gyromagnetic ratio of Kerr-Newman hairy black holes as a function of the frequency in blue, while the extremal hairy black hole solutions are in green.
As is clear from the figure, no solutions have a gyromagnetic ratio larger than two and the only solutions have have a ratio of two, are the hairless ones.
For a fixed frequency below the Hod point, $w_{\rm Hod}$, a maximal gyromagnetic ratio is obtained by the extremal hairy black hole.
However, as we were not able to continue the extremal hairy black hole spiral beyond the first branch of solutions, it is unclear how the envelope continues.

\newpage
\textit{Comments:} I suspect that our solutions can not be generated via the Ernst prescription and as such, the result of Reina and Treves\cite{Reina:1975rt} does not apply to them.
But I think that it is still interesting that our solutions then fall outside this class of axially symmetric stationary solutions due to having $g<2$.
What the meaning of this is, or whether it has any meaning at all, is not clear to me.
But I thought this series of papers by Ernst et al. might be relevant to this topic.

\begin{figure}[H]
  \begin{center}
    \includegraphics[width=0.79\textwidth]{w-g.pdf}
  \end{center}
  \caption{The gyromagnetic ratio as a function of the frequency. Hairy black hole solutions in blue and extremal solutions in solid green.}
  \label{fig:gyro}
\end{figure}

    
\bibliographystyle{h-physrev5.bst}
\bibliography{references}

\end{document}


