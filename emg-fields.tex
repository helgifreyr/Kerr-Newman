\documentclass[a4paper]{article}
\usepackage{haus}
\usepackage{a4wide}
\title{Visualizing the electromagnetic fields of hairy black holes}
\begin{document}
\newcommand\scalemath[2]{\scalebox{#1}{\mbox{\ensuremath{\displaystyle #2}}}}

% \section*{Kerr-Newman}
%
% We choose a basis $e^a$ which has the following dual basis 
%
% \begin{align*}
%   e_a^\mu &= \left(
%     \begin{array}{cccc}
%      \frac{a^2+r^2}{\sqrt{\Delta  \Sigma }} & 0 & 0 & \frac{a}{\sqrt{\Delta  \Sigma }} \\
%      0 & \sqrt{\frac{\Delta }{\Sigma }} & 0 & 0 \\
%      0 & 0 & \frac{1}{\sqrt{\Sigma }} & 0 \\
%      \frac{a \sin \theta}{\sqrt{\Sigma }} & 0 & 0 & \frac{\csc \theta}{\sqrt{\Sigma }} \\
%     \end{array}
%     \right).
% \end{align*}
%
% The electromagnetic potential for a Kerr-Newman black hole is
%
% \begin{align*}
% 	A_\mu &= \left\{\frac{Q r}{\Sigma },0,0,-\frac{a Q r \sin ^2\theta}{\Sigma }\right\},
% \end{align*}
% and we know that $F_{\mu\nu}=\partial_\mu A_\nu - \partial_\nu A_\mu$, so
%
% \begin{align*}
%   F_{\mu\nu} &=
% \left(
% \begin{array}{cccc}
%  \frac{a^2+r^2}{\sqrt{\Delta  \Sigma }} & 0 & 0 & \frac{a}{\sqrt{\Delta  \Sigma }} \\
%  0 & \sqrt{\frac{\Delta }{\Sigma }} & 0 & 0 \\
%  0 & 0 & \frac{1}{\sqrt{\Sigma }} & 0 \\
%  \frac{a \sin \theta}{\sqrt{\Sigma }} & 0 & 0 & \frac{\csc \theta}{\sqrt{\Sigma }} \\
% \end{array}
% \right)
% \end{align*}
%
% From this, we can express the Maxwell tensor in the tetrad above,
%
% \begin{align*}
%   F_{ab} &= F_{\mu\nu}e_a^\mu e_b^\nu \\
%   &= \left(
% \begin{array}{cccc}
%  0 & -\frac{Q \left(\cos (2 \theta ) a^2+a^2+2 r^2\right)}{2 \Sigma ^2} & \frac{a^2 Q r \sin (2 \theta )}{\sqrt{\Delta } \Sigma ^2} & 0 \\
%  \frac{Q \left(-\sin ^2\theta a^2+a^2+r^2\right)}{\Sigma ^2} & 0 & 0 & 0 \\
%  -\frac{a^2 Q r \sin (2 \theta )}{\sqrt{\Delta } \Sigma ^2} & 0 & 0 & -\frac{2 a Q r \cos \theta}{\Sigma ^2} \\
%  0 & 0 & \frac{2 a Q r \cos \theta}{\Sigma ^2} & 0 \\
% \end{array}
% \right)
% \end{align*}
%
% From this expression, we can read off the electric field:
%
% \begin{align*}
%   E_idx^i &= \left(-\frac{Q \left(a^2 \cos (2 \theta )+a^2+2 r^2\right)}{2 \Sigma ^2}\right)dr + \left(\frac{a^2 Q r \sin (2 \theta )}{\sqrt{\Delta } \Sigma^2}\right) d\theta,
% \end{align*}
%
% and the magnetic field
%
% \begin{align*}
%   B_idx^i &= \left(\frac{2 a Q r \cos \theta}{\Sigma ^2}\right)dr
% \end{align*}
%
% Some observations regarding these expressions:
%
% \begin{itemize}
%   \item Both $E$ and $B$ vanish when $Q=0$, as expected,
%   \item When $a=0$, $B$ vanishes and $E$ reduces to the Reissner-Nordström case,
%   \item Incidentally, in the equatorial plane, $\theta=\pi/2$, $B$ vanishes and $E$ reduces to the Reissner-Nordström case, just as for $a=0$, even if $a\neq0$. I don't see an obvious physical reason for why this happens.
% \end{itemize}

\section*{Kerr black holes with scalar hair}

As usual, we use the following metric ansatz for KBNsSH

\begin{align*}
   ds^2 = -e^{F_0}Ndt^2 + e^{2F_1}\left( \frac{dr^2}{N} + r^2d\theta^2 \right) + e^{2F_2}r^2\sin^2\theta \left(d\varphi - Wdt \right)^2.
\end{align*}
An obvious choice of tetrad is

\begin{align*}
  e^0 &\equiv e^{F_0}dt, \\
  e^1 &\equiv \frac{e^{F_1}}{\sqrt{N}}dr, \\
  e^2 &\equiv re^{F_1}d\theta, \\
  e^3 &\equiv r\sin\theta e^{F_2}(d\varphi-Wdt),
\end{align*}
whose dual basis is

\begin{align*}
	 e_0 &\equiv \frac{e^{-{F_0}}}{\sqrt{N}}(dt+Wd\varphi), \\
	 e_1 &\equiv  e^{-{F_1}} \sqrt{N}dr, \\
	 e_2 &\equiv \frac{e^{-{F_1}}}{r}d\theta, \\
	 e_3 &\equiv \frac{e^{-{F_2}} \csc \theta}{r}d\varphi.
\end{align*}

In order to calculate the stress-energy tensor of the Maxwell field, we use the electromagnetic potential for our hairy black holes

\begin{align*}
	A_\mu dx^\mu &= \left({A_t}-\frac{\sin \theta {A_\phi} W}{r^2}\right) dt + \sin\theta {A_\phi}d\varphi
\end{align*}
and since $F_{\mu\nu}=\partial_\mu A_\nu - \partial_\nu A_\mu$, we find that

\begin{align*}
  F_{\mu\nu} &= 
  \scalemath{0.6}{
  \left(
\begin{array}{cccc}
 0 & -\frac{2 {A_\phi} \sin \theta W}{r^3}+\frac{\sin \theta {A_\phi}_{,r} W}{r^2}-{A_t}_{,r}+\frac{{A_\phi} \sin \theta W_{,r}}{r^2} &
   \frac{{A_\phi} \cos \theta W}{r^2}+\frac{\sin \theta {A_\phi}_{,\theta} W}{r^2}-{A_t}_{,\theta}+\frac{{A_\phi} \sin \theta W_{,\theta}}{r^2} & 0 \\
 \frac{2 {A_\phi} \sin \theta W}{r^3}-\frac{\sin \theta {A_\phi}_{,r} W}{r^2}+{A_t}_{,r}-\frac{{A_\phi} \sin \theta W_{,r}}{r^2} & 0 & 0 & \sin \theta
   {A_\phi}_{,r} \\
 -\frac{{A_\phi} \cos \theta W}{r^2}-\frac{\sin \theta {A_\phi}_{,\theta} W}{r^2}+{A_t}_{,\theta}-\frac{{A_\phi} \sin \theta W_{,\theta}}{r^2} & 0 & 0 &
   {A_\phi} \cos \theta+\sin \theta {A_\phi}_{,\theta} \\
 0 & -\sin \theta {A_\phi}_{,r} & -{A_\phi} \cos \theta-\sin \theta {A_\phi}_{,\theta} & 0 
\end{array}
\right)}
\end{align*}

From this, we can express the Maxwell tensor in the tetrad given above,

\begin{align*}
  F_{ab} &= 
  \scalemath{0.5}{
  \left(
\begin{array}{cccc}
 0 & \frac{e^{-{F_0}-{F_1}} \left(-{A_t}_{,r} r^3-\left(r^2-1\right) \sin \theta
   W {A_\phi}_{,r} r+{A_\phi} \sin \theta \left(r W_{,r}-2 W\right)\right)}{r^3} & \frac{e^{-{F_0}-{F_1}} \left(-{A_t}_{,\theta}
   r^2-\left(r^2-1\right) \sin \theta W {A_\phi}_{,\theta}+{A_\phi} \left(\sin \theta
   W_{,\theta}-\left(r^2-1\right) \cos \theta W\right)\right)}{r^3 \sqrt{N}} & 0 \\
 \frac{e^{-{F_0}-{F_1}} \left({A_t}_{,r} r^3+\left(r^2-1\right) \sin \theta W {A_\phi}_{,r} r+{A_\phi} \sin \theta \left(2 W-r W_{,r}\right)\right)}{r^3} & 0 & 0 & \frac{e^{-{F_1}-{F_2}} \sqrt{N} {A_\phi}_{,r}}{r} \\
 \frac{e^{-{F_0}-{F_1}} \left({A_t}_{,\theta} r^2+\left(r^2-1\right) \sin \theta W {A_\phi}_{,\theta}+{A_\phi} \left(\left(r^2-1\right) \cos \theta W-\sin \theta
   W_{,\theta}\right)\right)}{r^3 \sqrt{N}} & 0 & 0 & \frac{e^{-{F_1}-{F_2}}
   \left({A_\phi} \cot \theta+{A_\phi}_{,\theta}\right)}{r^2} \\
 0 & -\frac{e^{-{F_1}-{F_2}} \sqrt{N} {A_\phi}_{,r}}{r} & -\frac{e^{-{F_1}-{F_2}} \left({A_\phi} \cot \theta+{A_\phi}_{,\theta}\right)}{r^2} & 0
\end{array}
\right)}
\end{align*}


From this expression, we can read off the electric field:

\begin{align*}
  E_i dx^i &= 
  \scalemath{0.75}{\left(\frac{e^{-{F_0}-{F_1}} \left(-\left(r^2-1\right) r \sin \theta {A_\phi}_{,r}
   W+\sin \theta {A_\phi} \left(r W_{,r}-2 W\right)+r^3
   \left(-{A_t}_{,r}\right)\right)}{r^3}\right)dr} \\
   &+ \scalemath{0.75}{\left(\frac{e^{-{F_0}-{F_1}} \left(-\left(r^2-1\right)
   \sin \theta {A_\phi}_{,\theta} W+{A_\phi} \left(\sin \theta W_{,\theta}-\left(r^2-1\right) \cos \theta W\right)+r^2 \left(-{A_t}_{,\theta}\right)\right)}{r^3
   \sqrt{N}}\right)d\theta}
\end{align*}

and the magnetic field

\begin{align*}
  B_idx^i &= \left(-\frac{\left({A_\phi}_{,\theta}+\cot \theta {A_\phi}\right) e^{-{F_1}-{F_2}}}{r^2}\right)dr \\ 
   &+ \left(\frac{\sqrt{N} {A_\phi}_{,r} e^{-{F_1}-{F_2}}}{r}\right)d\theta
\end{align*}

\end{document}


